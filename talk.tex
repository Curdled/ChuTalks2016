\documentclass{beamer} %
\usetheme{CambridgeUS}
\usepackage[latin1]{inputenc}
\usefonttheme{professionalfonts}
\usepackage{times}
\usepackage{amsmath}
\usepackage{verbatim}
\usepackage{pst-node}
\usepackage{tikz-cd}

\newcommand\blfootnote[1]{%
  \begingroup
  \renewcommand\thefootnote{}\footnote{#1}%
  \addtocounter{footnote}{-1}%
  \endgroup
}

\author{Joe Isaacs}
\title{Monads and Comonads}
\institute{Churchill CompSci talks}
\date{November 16, 2016}

\begin{document}

\begin{frame}
  \titlepage
\end{frame}

\section{Type Classes in Haskell}

\begin{frame}[fragile]\frametitle{Type classes}
\begin{verbatim}
class Eq a where
  (==) :: a -> a -> Bool
\end{verbatim}

\begin{verbatim}
instance Eq Integer where
  (==) a b = intEq a b

triEq :: (Eq a) => a -> a -> a -> Bool
triEq a b c = (a==b) && (a==c) && (b==c)
\end{verbatim}
\end{frame}

\section{Category theory background}

\begin{frame}{Category theory background}
  A category contains:
  \begin{itemize}
    \item Objects ($A,B,C,\dots$)

    \item Morphisms. ($f:B\rightarrow C$)

    \item Each object must have an identity morphism. ($id_A$ for an object $A$)

    \item Morphisms compose. ($f\circ g : A \rightarrow C$, if $g:A\rightarrow B$)
  \end{itemize}

  Categories must also follow three laws:

  \begin{itemize}
    \item $f \circ (g \circ h) = (f\circ g) \circ h$

    \item There must be a morphism $h:A\rightarrow C$ such
      that $h = f\circ g$


  \end{itemize}
  Each object $A$ must contain an identity morphism $\mathit{id}_A : A\rightarrow A$.
\end{frame}

\begin{frame}{$\mathit{Hask}$}
  \begin{itemize}
    \item We take objects $A,B$ to be types \texttt{A} and \texttt{B}

    \item We take morphisms $f : A\rightarrow B$ to be functions of type \mbox{\texttt{f :: a -> b}}

    \item Composition of morphisms $\circ$ will be function composition (written \texttt{(.)} in Haskell).

    \item $f \circ g$ goes to \texttt{(f . g) x} which in haskell in the same as \mbox{\texttt{f (g (x))}}.
    \end{itemize}
\end{frame}

\section{Functor}

\begin{frame}{Functor}
  \begin{itemize}
    \item Functors $F$ will map from a category $C$ to a category $D$ written as
      \[ F : C \rightarrow D \]

    \item An object $A$ in $C$ will be mapped to $F(A)$ in $D$

    \item (Covariant) functors map morphisms $f : A \rightarrow B$ to $F(f) : F(A) \rightarrow F(B)$

    \item $F(id_A) = id_{F(A)}$ where $id_A : A \rightarrow A$ and
      $id_{F(A)} : F(A) \rightarrow F(A)$

    \item $F(f \circ g) = F(f) \circ F(g) $ (functors distribute over morphism composition).
  \end{itemize}
\end{frame}

\begin{frame}[fragile]\frametitle{Functor in Haskell}
  \begin{itemize}
    \item A functor $F$ will map $f : A \rightarrow B$ to $F(f) : F(A) \rightarrow F(B)$

    \item
\begin{verbatim}
class Functor f where
   fmap :: (a -> b) -> f a -> f b
\end{verbatim}

  \item $id_A = \texttt{id :: a -> a}$

  \item $id_{F(A)} = \texttt{fmap id :: f a -> f a}$

  \item $F(h\circ g) = \texttt{fmap (h.g) :: f a -> f c}$

  \item $F(h) \circ F(g) = \texttt{fmap h . fmap g :: f a -> f c}$
  \end{itemize}
  where $\circ = \texttt{(.)}$, $h = \texttt{h :: b -> c}$,  $g = \texttt{g :: a -> b}$ and
  $F = \texttt{Functor f}$
\end{frame}

\begin{frame}[fragile]\frametitle{Functor example \texttt{Maybe}}
\begin{verbatim}
data Maybe a = Nothing | Just a
instance Functor Maybe where
  fmap f (Just x) = Just (f x)
  fmap _ Nothing  = Nothing
\end{verbatim}
\begin{itemize}
  \item Must check $F(id_A) = id_{F(A)}$
\begin{verbatim}
  fmap id x
case (x == Just a)
  fmap id (Just a) a ==> Just (id a) ==> Just a
case (x == Nothing)
  fmap id Nothing ==> Nothing

==> fmap id = id
\end{verbatim}
\end{itemize}
\end{frame}

\begin{frame}[fragile]\frametitle{Functor example \texttt{Maybe}}
Check $F(f \circ g) = F(f) \circ F(g)$
\begin{verbatim}
case (Just a)
      fmap (f.g) (Just a)
  ==> Just ((f.g) a)
  ==> Just (f (g a))

      (fmap f . fmap g) (Just a)
  ==> fmap f (fmap g (Just a))
  ==> fmap f (Just (g a))
  ==> Just (f(g a))
case (Nothing)
  fmap (f.g) Nothing == Nothing
      fmap f . fmap g Nothing
  ==> fmap f Nothing
  ==> Nothing
\end{verbatim}
\end{frame}

\section{Monad}

\begin{frame}{Monad}
  \begin{itemize}
    \item Defined by $(M,\mathit{join},\mathit{unit})$

    \item $M$ is also a functor

    \item $\mathit{join}$ is the transformation $M (M a) \rightarrow M a$ which satisfies
      \[ \mathit{join} \circ M (\mathit{join}) = \mathit{join} \circ \mathit{join}  \]
      and \[ \mathit{join} \circ M(M(f)) = M(f) \circ \mathit{join} \]

    \item $\mathit{unit}$ is the transformation $a \rightarrow M a$ which satisfies
      \[ \mathit{join} \circ M \mathit{unit} = \mathit{join} \circ \mathit{unit}  = id_M \]

      from $M \rightarrow M$ and \[ \mathit{unit} \circ f = M(f) \circ \mathit{unit} \]


  \end{itemize}
\end{frame}

\begin{frame}[fragile]\frametitle{Monad in Haskell}
We have seen the categorical definition of monads above, this could be a
Haskell implementation.
\begin{verbatim}
class Functor m => Monad m where
  unit :: a -> m a
  join  :: m (m a) -> m a
\end{verbatim}
\begin{itemize}
  \item $\mathit{unit} = \texttt{unit}$

  \item $\mathit{join} = \texttt{join}$
  \item The monad laws can now be written in Haskell as:
\begin{verbatim}
join . fmap join     = join . join
join . fmap unit     = join . unit    = id
unit . f             = fmap f . unit
join . fmap (fmap f) = fmap f . join
\end{verbatim}
\end{itemize}
\end{frame}

\begin{frame}[fragile]\frametitle{What do the laws mean?}
We can explain all the laws
with commutative diagrams, here is the first:

\[\mathit{join} \circ M(\mathit{join}) = \mathit{join} \circ \mathit{join}\]

\[
\begin{tikzcd}
M^3 \arrow{r}{\mathit{join}} \arrow[swap]{d}{M(\mathit{join})} & M^2 \arrow{d}{\mathit{join}} \\
M^2 \arrow{r}{\mathit{join}} & M
\end{tikzcd}
\]

When written in Haskell it looks like this.

\begin{verbatim}
join . fmap join = join . join
\end{verbatim}
\end{frame}

\begin{frame}[fragile]\frametitle{How Monads are really implemented in Haskell}
\begin{itemize}
  \item Haskell implements monad in different but equivalent way:

\begin{verbatim}
class Functor m => Monad m where
  return :: a -> m a
  (>>=)  :: m a -> (a -> m b) -> m b
  fail   :: String -> m a
\end{verbatim}
\item \texttt{(>>=)} (pronounced bind)
\item $\texttt{return} = \texttt{unit}$
\end{itemize}

\end{frame}

\begin{frame}[fragile]\frametitle{Joining bind}
\begin{itemize}

  \item  \texttt{join :: m m a -> m a}

  \item  \texttt{>>=  :: m a -> (a -> m b) -> m b}

\item \texttt{(>>=)} can be derived from \texttt{unit} and \texttt{fmap}:

\begin{verbatim}
join x = x >>= id  (id :: m a -> m a)

f :: a -> b
x :: m a

x >>= f = join (fmap f x)  (fmap f :: m m b)
\end{verbatim}
\end{itemize}
\end{frame}

\begin{frame}[fragile]\frametitle{Monad laws using \texttt{(>>=)}}
  \begin{itemize}
    \item
\begin{verbatim}
m >>= return = m
return m >>= f = f m
(m >>= f) >>= g = m >>= (\x -> f x >>= g)
\end{verbatim}
These are left identity, right identity and associativity
  \item It can also be show that these three monad laws are equivalent to the four monad laws
    stated previously.
\end{itemize}
\end{frame}

\begin{frame}[fragile]\frametitle{Example: \texttt{Maybe}}
\begin{verbatim}
instance Monad Maybe where
  return x      = Just x
  fail _        = Nothing
  Just x  >>= f = f x
  Nothing >>= _ = Nothing
\end{verbatim}
The Maybe instance of Monad must also follow the monad laws stated previously.
\begin{verbatim}
m >>= return = m

Just x >>= return ==> return x ==> Just x

Nothing >>= return ==> Nothing
\end{verbatim}
\end{frame}

\section{Monad uses}

\begin{frame}[fragile]\frametitle{Computations with possible errors}
\begin{verbatim}
computeB :: a -> Maybe b
computeC :: b -> Maybe c

compute  :: a -> Maybe c
compute a = case computeB of
              Just b  -> computeC b
              Nothing -> Nothing
\end{verbatim}
\end{frame}

\begin{frame}[fragile]\frametitle{Computations with possible errors Cont.}
\begin{verbatim}
computeB :: a -> Maybe b
computeC :: b -> Maybe c

compute  :: a -> Maybe c
compute a = computeB a >>= computeC
\end{verbatim}
\end{frame}

\begin{frame}[fragile]\frametitle{do-notation}
\begin{verbatim}
do x                ==> x

do let y = z        ==> let y = z in do x
   x

do y <- z           ==> z >>= \y -> do x
   x

do y                ==> y >>= \_ -> do x
   x

\end{verbatim}
\end{frame}

\begin{frame}[fragile]\frametitle{Using do-notation}
\begin{verbatim}
compute :: a -> Maybe c
compute a =
  do b <- computeB a
     c <- computeC b
     return c
\end{verbatim}

Looks a lot like imperative code
\end{frame}




\begin{frame}[fragile]\frametitle{IO with monads}
  \begin{itemize}

    \item IO monad, with two operations
      \begin{itemize}
        \item \texttt{putStr :: String -> IO ()} builds a monad that when run will
          print the string passed into \texttt{putStr} and return
        \item \texttt{getLine :: IO String}, build a monad that when run
          will read a stream of characters for the keyboard up to a newline.
      \end{itemize}
    \item This means the IO monad represents a sequence of computations, which are order in the same order
      as the binds.

    \item The IO monad will then be evaluated once it is returned to the Haskell runtime or \texttt{unsafePerformIO :: IO a -> a} is performed.

  \end{itemize}
\end{frame}


\begin{frame}[fragile]\frametitle{Example IO}
  \begin{verbatim}
    main :: IO ()
    main = do x <- getLine
              putStr ("return: " ++ x)
  \end{verbatim}
\end{frame}

\begin{frame}[fragile]\frametitle{\texttt{>>=} for IO}
\begin{verbatim}
instance Monad IO a where
  return a       = IO (\s -> (s,a))
  (IO m) (>>=) f = IO
      (\s -> case MUTABLE(m s) of (m',_) -> f m')
\end{verbatim}
\end{frame}

\section{Comonad}

\begin{frame}[fragile]\frametitle{Comonad}
  \begin{itemize}
    \item Dual of a monad


    \item Remember Monad has two morphisms
      \begin{itemize}
        \item $\mathit{unit} : a \rightarrow M a$
        \item $\mathit{join} : M (M a) \rightarrow M a$
      \end{itemize}

    \item A comonad $C$ defines two transformations:
      \begin{itemize}
        \item $\mathit{extract} : C a \rightarrow a$

        \item $\mathit{duplicate}  : C a \rightarrow C (C a)$
      \end{itemize}
    \item Satisfying three laws:
      \begin{itemize}
        \item $\mathit{extend} \circ \mathit{duplicate} = I_C$
        \item $C(\mathit{extract}) \circ \mathit{duplicate} = I_C$
        \item $C(\mathit{duplicate}) \circ \mathit{duplicate} = \mathit{duplicate}
          \circ \mathit{duplicate}$
      \end{itemize}
    \item In the triple $(C,\mathit{extract},\mathit{duplicate})$, $C$ is a functor.
  \end{itemize}
\end{frame}

\begin{frame}[fragile]\frametitle{Comonad in Haskell}
\begin{verbatim}
class Functor w => Comonad w where
  extract :: w a -> a
  extend  :: (w a -> b) -> w a -> w b
\end{verbatim}
since
\begin{verbatim}
duplicate :: m a   -> m m a
fmap f    :: m m a -> m b
\end{verbatim}
We can define extend as \texttt{extend f = (fmap f) . duplicate}

A Comonad must also obey these laws:
\begin{verbatim}
extend extract      = id
extract . extend f  = f
extend f . extend g = extend (f. extend g)
\end{verbatim}
\end{frame}

\begin{frame}[fragile]\frametitle{CArray and associated functions}
\begin{verbatim}
data CArray i a = CA (Array i a) i 

indices :: Array i a -> [i]

bounds :: Array i a -> [i]

array :: [i] -> [a] -> Array i a

(!) :: Array i a -> a

(?) :: CArray i a -> a   (a safe ! with an offset)
\end{verbatim}
\end{frame}

\begin{frame}[fragile]\frametitle{Array filters are comonads}

\begin{verbatim}
instance Comonad (CArray i) where
  extract (CA arr c) = arr!c
  extend f (CA x c)  =
    let  es' = map (\i -> (i,f(CA x i))) (indices x)
    in   CA (array (bounds x) es') c

laplace1D :: Num a => CArray Integer a -> a
laplace1D a = (a ? (-1)) + (a ? 1)  - 2 * (a ? 0)
\end{verbatim}
\texttt{(?)} will try to get the value at \texttt{i+i'} of the value \texttt{a}
where \texttt{(CA a i)}

\blfootnote{Example from `A Notation for Comonads' by Dominic Orchard and Alan Mycroft}
\end{frame}

\begin{frame}
  \center \Huge Questions?
\end{frame}

\section{Monadic Transformers}


\begin{frame}[fragile]\frametitle{Mixing Monads}
  \begin{itemize}
    \item What happens if we want to have both IO and logging in the same functions?

    \item Use monad transformers.

    \item Monad transformers require another function to be defined \mbox{\texttt{lift :: m a -> t m a}}.

    \item This must satisfy the laws:
      \begin{itemize}
        \item \texttt{lift . return = return}

        \item \texttt{lift (m >>= f) = lift m >>= (lift . f)}
      \end{itemize}

    \item This is captured in Haskell by:
\begin{verbatim}
class MonadTrans t where
  lift :: m a -> t m a
\end{verbatim}
  \end{itemize}
\end{frame}

\begin{frame}[fragile]\frametitle{Logging with IO}
\begin{verbatim}
type LoggingIO l a = WriterT l IO a

log :: String -> LoggingIO [String] ()
log s = tell [s]

example :: LoggingIO [String] ()
example = do log "Print 1"
             lift (print 1)

main = do str <- execWriterT example
          (print str)
\end{verbatim}
\end{frame}

\begin{frame}[fragile]\frametitle{We cannot just compose any Monad}
\begin{itemize}
  \item Remember the type of \texttt{join :: M(M a) -> M a}

  \item If we extend this to monad transformers we get \[\texttt{join :: M(N(M(N a))) -> M(N a)}\]

  \item We cannot infer the \texttt{join} method for transformers just from the \texttt{join} functions for
        the transformer and the monad being transformed.

  \item To allow composition we must also have a function \[\texttt{distrib :: t (m a) -> m (t a)}\]
\end{itemize}
\end{frame}

\begin{frame}[fragile]\frametitle{Monoid}
    A monoid is an object $M$ with two morphisms $\oplus: M \times M \rightarrow M$ and $\mathit{unit} :M$.

    Where both \[\mathit{unit} \oplus M = M = M \oplus \mathit{unit} \text{ (left and right
        identity)}\] and
    \[ M \oplus (M \oplus M) = (M \oplus M)
      \oplus M \text{ (associativity)} \]

\begin{verbatim}
class Monoid a where
  mzero   :: a
  mappend :: a -> a -> a
\end{verbatim}
$\mathit{unit} = \texttt{mzero}$ and $\oplus = \texttt{mappend}$
\end{frame}

\begin{frame}[fragile]\frametitle{Writer Monad Motivation: Logging}
  \begin{itemize}
    \item Want to store order set of stages throughout a computation.
\begin{verbatim}
addOne :: Int -> Int -> [String]
addOne x = (x+1,["Added 1 to " ++ show x])

applyLogger :: Monoid m =>
               (v,m) -> (v -> (o, m)) -> (o,m)
applyLogger (input,log) f = (o, log `mappend` l)
  where (o,l) = f input

addOne 0 `applyLogger` addOne
`applyLogger` addone

> (3,["Added 1 to 0"
     ,"Added 1 to 1","Added 1 to 2"])
\end{verbatim}
  \end{itemize}
\end{frame}

\begin{frame}[fragile]\frametitle{Write Monad}
\begin{verbatim}
newtype Writer w a = Writer { runWriter :: (a,w) }

instance (Monoid w) => Monad (Writer w) where
  return x = writer (x,mempty)
  (Writer (x,v)) >>= f = let (Writer (y,v')) = f x
                         in Writer (y,v `mappend` v')
\end{verbatim}
\end{frame}

\begin{frame}[fragile]\frametitle{Logging with the Write Monad}
\begin{verbatim}

logNumber :: Int -> Writer [String] Int
logNumber x = Writer (x, ["Recorded " ++ (show x))

runWriter
(do a <- logNumber 10
   b <- logNumber 11
   return (a*b))
> (110,["Recorded 10", "Recorded 11"])
\end{verbatim}
\end{frame}


\end{document}
